\documentclass[colorlinks=true,pdfstartview=FitV,linkcolor=blue,
            citecolor=red,urlcolor=magenta]{ligodoc}

\usepackage{graphicx}
\usepackage{amssymb}
\usepackage{amsmath}
\usepackage{longtable}
\usepackage{rotating}
\usepackage[usenames,dvipsnames]{color}
\usepackage{fancyhdr}
\usepackage{subfigure}
\usepackage{hyperref}
\ligodccnumber{T}{11}{XXXXX}{}{vX}% \ligodistribution{AIC, ISC}


\title{}

\author{Jacob Bernhardt}

\begin{document}

% How long should the plan be, and what should be included?
% A research plan of two or three pages, carefully thought out and precisely worded, should be sufficient to make all the important points. Concerning structure and content: start out with the sections indicated below and try to answer the questions provided in each part. When you have this material developed, you may be able to reorganize it so that it flows more logically while covering the same ground.

\section{Introduction} % Introduction/Background
% What is the general technical area in which you will be working? What is the problem that you are trying to solve, and how did the problem arise? Why is its solution interesting or worthwhile? What is the status of related research by your mentor or by the group that you will be joining, and what will be the contribution and significance of your effort if it is successful?
% 
% You will probably have to ask your mentor a lot of questions and read some or all of the reference material provided for you in order to answer these questions and others below.


\section{Objectives}
% What do you aim to accomplish in your project? What will you measure, and under what conditions; or, what will you calculate, model, or simulate; or what will you design, and what are the requirements; or what will you build or test? What is your starting point? What are your initial assumptions or conditions? What will be the result or product of a successful outcome for your project? What are the criteria for project completion or for success? (In other words, how will you know when you have accomplished what you set out to do?)


\section{Approach}
% Specifically, how will you reach your objective or produce your desired final product? What are the principal steps or milestones along the path? How long will each take? What steps promise to be the most difficult, and how will you overcome the difficulties? What equipment or other resources will you need? Which of these are inherited, and which will you have to make or procure? With what other people or groups will you be collaborating? Will completion of your project depend on results from other people in related projects? (That question may be especially pertinent for team projects.)


\section{Project Schedule}
% Preparing a schedule of the principal activities and events is a good way of showing the readers that you have taken a systematic approach to planning your work.


% References
% List all pertinent papers or reports that you have consulted to prepare your plan. Include remarks or suggestions from your prospective supervisor, from graduate students, or from other people with whom you have talked.

\end{document}
 
